\documentclass[
12pt,
spanish,
singlespacing,
parskip,
headsepline,]{article}
\usepackage{inputenc}
\usepackage[spanish]{babel}
\usepackage{palatino}

%opening
\title{Monitoreo y control de radares ATSA}
\author{Gonzalo Nahuel Vaca}

\begin{document}

\maketitle

\begin{abstract}

Este documento es una propuesta de proyecto en el marco de la Maestría en Internet de las Cosas de la Universidad de Buenos Aires.

\end{abstract}

\newpage

\tableofcontents

\newpage

\section{Propósito del proyecto}

La Universidad de Buenos Aires (UBA) solicita que sus candidatos de Maestría realicen un proyecto que consuma 600 horas de ingeniería.
Las características técnicas que se esperan son las siguientes:

\begin{itemize}
	\item Percepción: programación y diseño de dispositivos embebidos.
	\item Transporte: uso de protocolos de comunicación.
	\item Procesamiento: persistencia y análisis de datos.
	\item Aplicación: interfaz gráfica de usuario.
	\item Negocio: gestión y monitoreo del sistema.
\end{itemize}

Las dificultades que enfrenta el sistema de radares se pueden solucionar con un proyecto que cumpla los requerimientos de UBA.
A continuación se enumeran los problemas identificados:

\begin{itemize}
	\item Persistencia: la arquitectura de datos no es adecuada para el volumen de información manejado, no es escalable y no es posible ofrecer un servicio profesional de análisis de datos.
	\item Comunicación con radares: los puntos de agregación funcionan con una colección de aplicaciones antiguas que no están correctamente integradas al sistema operativo.
	\item Interfaz gráfica: su aspecto es antiguo, no es apta para \emph{smartphones} y su funcionalidad es lenta y limitada.
	\item Dependencia de \emph{software} sin licencias: la solución actual carece de soporte en puntos críticos y aumenta la superficie de posibles fallas.
\end{itemize}

\section{Descripción técnica-conceptual del proyecto a realizar}

\section{Alcance del proyecto}

El proyecto incluye en su alcance la creación de un servicio embebido que recoja la información de los radares \emph{Exemis}.
Ofrecerá telemetría \emph{GSM} para realizar acceder a la Internet.
Además, se deberá resolver la comunicación con un servidor que forma parte del desarrollo.
El servidor ofrecerá una interfaz de programación de aplicaciones (API) que permitirá el monitoreo y control de los radares.

Se creará una arquitectura de persistencia moderna que permita la gestión de grandes volúmenes de datos.


\section{Supuestos del proyecto}

Para el desarrollo del presente proyecto se supone que se tendrá acceso a la documentación necesaria para desarrollar implementar una comunicación con los radares Exemys.
Además se espera tener acceso al sistema actual con la finalidad de portarlo como \emph{legacy}.

\section{Requerimientos}

\section{Entregables principales del proyecto}

\section{Desglose del trabajo en tareas}

\section{Diagrama de Activity On Node}

\section{Diagrama de Gantt}

\section{Matriz de asignación de responsabilidades}

\section{Gestión de riesgos}

\section{Gestión de calidad}

\section{Comunicación del proyecto}

\section{Seguimiento y control}

\section{Procesos de cierre}

Establecer las pautas de trabajo para realizar una reunión final de evaluación del proyecto, tal que contemple las siguientes actividades:

\begin{itemize}
	\item El responsable del proyecto analizará si se respetó el plan de proyecto original, comparando dicho cronograma contra el real e identificando las tareas que mayor divergencia de tiempo presentaron y sus causas.
	\item El responsable del proyecto realizará una lista de las técnicas y procedimientos que le resultaron útiles para cumplir con los objetivos preestablecidos, y las que le hayan generado retrasos, indicando las posibles causas.
	\item El responsable del proyecto se encargará de agradecer a todas las personas involucradas en el proyecto.
\end{itemize}

\end{document}
